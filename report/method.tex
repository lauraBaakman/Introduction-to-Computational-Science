%!TEX root = report.tex

\todo[inline]{Intro plus what one can find in this section... }
When simulating the dynamics of cracking material, one needs a model that represents the properties of that materials, i.e., in the case of this paper mud or paint. In \cref{ss:method:model} we firstly present our model. To be able to show the contrast between the models, in \cref{ss:method:contrast}, we shortly review the model given by \citeauthor{vogel2005studies2} in \cref{ss:method:vogel}.

\subsection{The model}\label{ss:method:model}

\begin{figure}
	\centering
	\resizebox{0.9\columnwidth}{!}{%
		\singleSpring
	}
	\caption{Illustration of a single connection (spring) between two particles.}
	\label{fig:3:}
\end{figure}

\todo[inline]{Decribe our model:  1. Properties of single particles and connection between two (springs)}

\todo[inline]{Describe our model: 2. Initialization of the grid and spring constants. (Which distribution and parameters)}

\todo[inline]{Describe how the model can be solved/stabilized}

\todo[inline]{Decribe how springs break (two or three methods)}

\todo[inline]{Reference that the step after this is the solve/stabilization part of this section.}

\subsection{The Vogel model}\label{ss:method:vogel}

\todo[inline]{Short review of the model from Vogel et al. So we can contrast this in the last section}

\subsection{Contrast}\label{ss:method:contrast}

\todo[inline]{Describe the important differences of the two models.}


\todo[inline]{Story from the presentation. Different types of particles. Springs. How the particles are connected, with those springs (square uniform grid). Things like energy and what is a stable system?}

