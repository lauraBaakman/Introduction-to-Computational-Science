%!TEX root = report.tex

\todo[inline]{Intro plus what one can find in this section... }
When simulating the dynamics of cracking material, one needs a model that represents the properties of that materials, i.e., in the case of this paper mud or paint. In \cref{ss:method:model} we firstly present our model. To be able to show the contrast between the models, in \cref{ss:method:contrast}, we shortly review the model given by \citeauthor{vogel2005studies2} in \cref{ss:method:vogel}.

\subsection{The model}\label{ss:method:model}

\laura{Read the following paragraph(s) which should implement the following todo: ``Decribe our model:  1. Properties of single particles and connection between two (springs)''}
To model the rupture dynamics of mud we opted for a spring simulation. In this simulation Mud is modeled by particles connected by springs. The cracking of mud is a very slow process and therefore we do not take the velocity and frictional forces into consideration, i.e., we ignore Newton's laws of motion and Stokes' law, which describes friction. We can therefore keep the particles massless and the simulation process simple, by only considering the locations of the particles. 

\begin{figure}
	\centering
	\singleSpring
	\caption{Illustration of a single connection, i.e., a spring, between two particles. The distance between particle $v_i$ and $v_j$ is given by $X$. The variable $l_0$ is used to denote the natural length of a spring with spring constant $k_0$.}
	\label{fig:method:spring}
\end{figure}

The image in \cref{fig:method:spring} shows the illustration of a single connection between two particles $v_i$ and $v_j$. The connection, as stated in the previous paragraph, is modeled using a spring. The force on a Hookean spring is given by $F = -k X\,$, where $F$ is restoring force exerted by the spring on whatever is pulling on the other end; and $X = |x_i - x_j|$ the distance between the two particles. If the stress on a spring exceeds some threshold, or is among the $k$ springs with the highest stress (see \cref{s:implementation}), it breaks and cracks appear.

\rick{Write the following paragraph(s) such that they implement the following todo: ``Describe our model: 2. Initialization of the grid and spring constants. (Which distribution and parameters)''}

The previous paragraphs described a single connection between particles. This is not enough to simulate an area of rupturing mud. To simulate this we opted to layout the particles in a uniform square grid. The particles all have a valency of four, where their neighbors are, in the beginning, for every free particle... \todo[inline]{(fuck forgot to introduce the difference between free and fixed particles. )}

\begin{figure}
	\centering
	\missingfigure{\plaatje{Illustration of a uniform square grid. There is no need for particle names and that kind of stuff, just a visualization of the uniform square grid, with a reasonable number of particles.}}
	\caption{Starting layout of the particles representing uncracked mud.}
	\label{fig:model:layout}
\end{figure}

\todo[inline]{Describe how the model can be solved/stabilized}

\todo[inline]{Decribe how springs break (two or three methods)}

\todo[inline]{Reference that the step after this is the solve/stabilization part of this section.}

\subsection{The Vogel model}\label{ss:method:vogel}

\todo[inline]{Short review of the model from Vogel et al. So we can contrast this in the last section}

\subsection{Contrast}\label{ss:method:contrast}

\todo[inline]{Describe the important differences of the two models.}


\todo[inline]{Story from the presentation. Different types of particles. Springs. How the particles are connected, with those springs (square uniform grid). Things like energy and what is a stable system?}

