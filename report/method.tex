%!TEX root = report.tex

\todo[inline]{Intro plus what one can find in this section... }
When simulating the dynamics of cracking material, one needs a model that represents the properties of that materials, mud in our case. In \cref{ss:method:model} we present our model. \Cref{ss:method:vogel} presents the model introduced by \citeauthor{vogel2005studies2} on which ours is loosely based. So that we can contrast the two models in \cref{ss:method:contrast}.

\subsection{The model}\label{ss:method:model}
% \todo[inline]{``Decribe our model:  1. Properties of single particles and connection between two (springs)''}
To model the rupture dynamics of mud we opted for a spring simulation. In this simulation mud is modeled by particles connected by Hookean springs. The cracking of mud is a very slow process and therefore we do not take the velocity and frictional forces into consideration, i.e., we ignore Newton's and Stokes' laws. Therefore we can use massless particles and only take the location of a particle into account. As a consequence it is possible to find the new position of particles using a matrix vector equation, which can be solved with a linear solver.

\begin{figure}
	\centering
	  \begin{tikzpicture}
	    \node [free]  (left)  at  (0,0)   {\freeParticle{i}};
	    \node [free]  (right)   at  (4,0) {\freeParticle{j}};
	    \drawSpring{left}{right}{0}

	    \node (leftUpper) at (0,0.5) {};
	    \node (rightUpper) at (4,0.5) {};
	    \draw[dashed, <->] (leftUpper)--(rightUpper) node[label] {$D_{ij} = \abs{x_i - x_j}$};

	    \node (leftLower) at (1,-0.5) {};
	    \node (rightLower) at (3,-0.5) {};
	    \draw[dashed, <->] (leftLower)--(rightLower) node[label, below=1ex] {$l_0$};
	  \end{tikzpicture} %
	\caption{Illustration of a spring, \spring{0}, connecting the particles \freeParticle{i} and \freeParticle{j}. The distance between the two particles is represented by $D_{ij}$, $x_i$ and $x_j$ refer to the position of \freeParticle{i} and \freeParticle{j}, respectively. The natural length of a spring is represented by $l_0$.}
	\label{fig:method:spring}
\end{figure}


% \todo[inline]{Decribe how springs break (two or three methods)}

\Cref{fig:method:spring} illustrates a single connection between two particles \freeParticle{i} and \freeParticle{j}. The connection between the two particle is modeled with the spring \spring{0}. The force $F$ on a Hookean spring is defined as
\begin{equation}\label{eq:method:hookeslaw}
	F = -k X,	
\end{equation}
where $F$ is the restoring force exerted by the spring on whatever is pulling on the other end. $X$ is the displacement of the string from it's relaxed position. To ensure that the final system of equations is linearly solvable we have to use springs with a natural length of zero. Consequently $X$ is equal to the distance between the particles that are connected by the spring, i.e., $D_{ij}$ in \cref{fig:method:spring}. If the stress on a spring exceeds some threshold, or is among the $n$ springs with the highest stress the string breaks, and cracks appear.

\laura{Read the following paragraph(s) which should implement the following todo: ``Describe our model: 2. Initialization of the grid and spring constants. (Which distribution and parameters)''}

In the previous paragraphs we described a single connection between particles. An area of rupturing mud can not be modeled using only a single connection. To simulate an area of mud we opted to layout multiple particles in a uniform square grid. An illustration of this initial layout is shown in \cref{fig:model:layout}. In this illustration the larger blue squares on the edge of the layout are, the so called the fixed particles. The free or movable particles are illustrated by the smaller red squares, each connected by a spring with its four neighbors. 
%
\begin{figure}
	\centering
	% \missingfigure{\plaatje{Illustration of a uniform square grid. There is no need for particle names and that kind of stuff, just a visualization of the uniform square grid, with a reasonable number of particles.}}
	\includegraphics[width=0.9\columnwidth]{img/uniform_square_grid.png}
	\caption{Starting layout of the particles representing uncracked mud. Image made using the software which was written for this paper.}
	\label{fig:model:layout}
\end{figure}
%
\rick{Write stuff about forces on particles and that we simplify them}
The force on a single particle is then given by:
%
\begin{equation}\label{eq:single:force}
	F_i = - \Sigma_{j \in N_i} k_{ij} \frac{x_i - x_j}{|x_i - x_j|}[|x_i - x_j| - l_0].
\end{equation}
%
The set $N_i$ are the neighboring particles of particle $v_i$. For every spring between two particles $v_i$ and $v_j$ a constants $k_ij$ can be chosen randomly or from some distribution, see \cref{s:implementation} for the discussion about the details. The variable $l_0$ represents the natural length of a spring, i.e., the length of a spring that is only connected to a single particle and thus is not stretched. To be able to linearly solve the system we discard this term by setting it to zero. By doing so the formula for the force $F_i$ on a particle $v_i$ is given by:
%
\begin{equation}\label{eq:single:force:simple}
	F_i = - \Sigma_{j \in N_i} k_{ij}(x_i - x_j).
\end{equation}

\rick{Describe how the model can be solved/stabilized}

With the formula in \eqref{eq:single:force:simple} we can now compute all the forces on every particle and move the particle to the locations, such that the system is stable, i.e., $F_i = 0 \forall_i$. With this we can find the $L$ matrix:
%
\begin{equation}\label{eq:a:}
	L = A^T K A .* C
\end{equation}
%
\begin{equation}\label{eq::}
	L\vec{x} = \vec{r}.
\end{equation}
%
If this formula we find the new positions $\vec{x}$ of the particles by solving the 

\todo[inline]{Reference that the step after this is the solve/stabilization part of this section.}

\subsection{The Vogel model}\label{ss:method:vogel}

\todo[inline]{Short review of the model from Vogel et al. So we can contrast this in the last section}

\subsection{Contrast}\label{ss:method:contrast}

\todo[inline]{Describe the important differences of the two models.}


\todo[inline]{Story from the presentation. Different types of particles. Springs. How the particles are connected, with those springs (square uniform grid). Things like energy and what is a stable system?}

