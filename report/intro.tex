%!TEX root = report.tex


\todo[inline]{Why is looking at the rupture dynamics of mud/paint useful. Give a nice example, e.g., the Mona Lisa.}

\todo[inline]{How is this problem tackled before. E.g., Vogel et al. which models the rupture dynamics of mud in the following way: ... -> conclusion a slow but representative model/method.}

\todo[inline]{Global idea of the model/method presented in this paper. ... -> hypothesis: fast but maybe not as a representative model and simulation of the crack dynamics of mud/paint.}

\todo[inline]{Structure of the paper -> stuff like section x discusses blaat. etc. you know ;)}

% \begin{figure}
% 	\centering
% 	\resizebox{0.9\columnwidth}{!}{%
% 		\initialGrid
% 	}
% 	\caption{Caption here}
% 	\label{fig:1:}
% \end{figure}

% \begin{figure}
% 	\centering
% 	\resizebox{0.9\columnwidth}{!}{%
% 		\stabilizedInitialGrid
% 	}
% 	\caption{Caption here}
% 	\label{fig:2:}
% \end{figure}

% \begin{figure}
% 	\centering
% 	\resizebox{0.9\columnwidth}{!}{%
% 		\cutGrid
% 	}
% 	\caption{Caption here}
% 	\label{fig:3:}
% \end{figure}

% \begin{figure}
% 	\centering
% 	\resizebox{0.9\columnwidth}{!}{%
% 		\stabilizedCutGrid
% 	}
% 	\caption{Caption here}
% 	\label{fig:4:}
% \end{figure}
