%!TEX root = report.tex
This section discusses our implementation of our simulation, based on the theory discussed in \cref{s:method}. We used C++ to simulate the dessication process and  Qt\cite{qt} and OpenGL\cite{openGL} to visualize it. \Cref{s:implementation:init} discusses how we initialized our grid. The next section explains how we have created the matrices and vectors in \cref{eq:method:stabilizationEquation,eq:method:stabilizationEquationLHS}.

\subsection{Initialize}
\label{s:implementation:init}

\todo[inline]{Refer to figure}

\todo[inline]{Init grid: refer to figure, verschil tussen fixed and free, minimaal aantal fixed particles}

\todo[inline]{Init particle: location determined by position in grid, otherwise no properties}

\todo[inline]{Init }


\todo[inline]{Hoe werkt init, leg uit hoe die matrices gebouwd worden}

\subsection{Stabilize}
\label{s:implementation:stabilize}


\todo[inline]{Noemen welke lib, waarom die, wat we ervan gebruikt hebben.}

\begin{equation}\label{eq:implementation:forceOnVo}
	\begin{split}
	F_{\freeParticle{0}} 	& = - [				\spring{0} \left(\fixedParticle{0} - \freeParticle{0}\right) +
												\spring{3} \left(\fixedParticle{3} - \freeParticle{0}\right) + \\
							& \quad\quad\;\;	\spring{4} \left(\freeParticle{1} - \freeParticle{0}\right)  +
												\spring{7} \left(\freeParticle{3} - \freeParticle{0}\right) ]
	\end{split}
\end{equation}


\begin{equation}\label{eq:implementation:forceOnVoVersimpled}
	\begin{split}
	\spring{0}\fixedParticle{0} + \spring{3}\fixedParticle{3} 
		&= - \spring{4}\freeParticle{1} - \spring{7}\freeParticle{3} + \\
		& \quad\: \freeParticle{0} \left( \spring{0} + \spring{3} + \spring{4} + \spring{7} \right)
	\end{split}
\end{equation}

\begin{equation}\label{eq:implementation:rhs}
{
	\arraycolsep=1pt
	\begin{bmatrix}
		\spring{0 + 3 + 4 + 7}	& -\spring{4}			& \cdots & 0\\
		- \spring{4} 			& \spring{1 + 4+ 5 + 8}	& \cdots & 0\\
		\vdots					& \vdots 				& \ddots & \vdots\\
		0						& 0 					& \cdots & \spring{16 + 19 + 20 + 23}\\
	\end{bmatrix}
}
\end{equation}
where $\spring{a + b} = \spring{a} + \spring{b}$. 

\begin{equation}\label{eq:implementation:adjacency}
	\begin{bmatrix}
		1 		& 0 		& 0 		& \cdots 	& 0\\
		0 		& 1 		& 0 		& \cdots 	& 0\\
		0 		& 0 		& 1 		& \cdots 	& 0\\
		1 		& 0 		& 1 		& \cdots 	& 0\\
		1 		& 1 		& 1 		& \cdots 	& 0\\
		\vdots	& \cdots 	& \cdots 	& \ddots 	& \vdots\\
		0 		& 0 		& 0 		& \cdots 	& 1\\
	\end{bmatrix}
\end{equation}

\begin{equation}\label{eq:implementation:springConstantMatrix}
	\begin{bmatrix}
		\spring{0} 	& 0 			& \cdots & 0\\
		0 			& \spring{1}	& \cdots & 0\\	
		\vdots 		& \vdots		& \ddots & \vdots\\
		0 			& \cdots		& \cdots & \spring{23}\\
	\end{bmatrix}
\end{equation}

\begin{equation}\label{eq:implementation:correctionMatrix}
	\begin{bmatrix}
		1 			& -1 			& \cdots & -1\\
		-1 			& 1				& \cdots & -1\\	
		\vdots 		& \vdots		& \ddots & \vdots\\
		-1 			& -1			& \cdots & 1\\		
	\end{bmatrix}
\end{equation}

\todo[inline]{Hoe werkt stabilize}

\subsection{Break Springs}
\label{s:implementation:break}
\todo[inline]{Hoe werkt break: denk dat deze sectie eruit gaat, zo boeiend is break niet.}

\begin{figure}
	\centering
	\resizebox{0.9\columnwidth}{!}{%
		\initialGrid
	}
	\caption{Caption here}
	\label{fig:1:}
\end{figure}

% \begin{figure}
% 	\centering
% 	\resizebox{0.9\columnwidth}{!}{%
% 		\stabilizedInitialGrid
% 	}
% 	\caption{Caption here}
% 	\label{fig:2:}
% \end{figure}

% \begin{figure}
% 	\centering
% 	\resizebox{0.9\columnwidth}{!}{%
% 		\cutGrid
% 	}
% 	\caption{Caption here}
% 	\label{fig:3:}
% \end{figure}

% \begin{figure}
% 	\centering
% 	\resizebox{0.9\columnwidth}{!}{%
% 		\stabilizedCutGrid
% 	}
% 	\caption{Caption here}
% 	\label{fig:4:}
% \end{figure}