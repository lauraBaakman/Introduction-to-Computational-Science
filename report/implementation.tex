%!TEX root = report.tex
This section discusses our implementation of our simulation, based on the theory discussed in \cref{s:method}. We used C++ to simulate the dessication process and  Qt\cite{qt} and OpenGL\cite{openGL} to visualize it. \Cref{s:implementation:init} discusses how we initialized our grid. The next section explains how we have created the matrices and vectors in \cref{eq:method:stabilizationEquation}.

\subsection{Initialize}
	\label{s:implementation:init}
	\begin{figure}
		\centering
		\resizebox{0.9\columnwidth}{!}{%
			\initialGrid
		}
		\caption{The square uniform grid as it is initialized. The dots labeled with \fixedParticle{0} through \fixedParticle{23} are fixed. The circles named \freeParticle{0} through \freeParticle{8} are free. The lines labeled \spring{0} through \spring{23} are springs with spring constants \spring{i}, for $i = 1\ldots23$.}
		\label{fig:implementation:intitialGrid}
	\end{figure}

	\Cref{fig:implementation:intitialGrid} presents the square uniform grid as it is initialized. 
	%Free
	The user selects the number of free particles requested in the grid. This number is then rounded up to get the number of particles required for a square grid. The location of the free particles are determined by the shape of the initial grid.
	% Fixed
	The fixed particles are placed  in such a way that their number is minimal and that every free particle is connected to four other particles. 
	%Springs
	The spring constants are sampled form a normal distribution whose mean and standard deviation are set by the user. 

\subsection{Stabilize}
	\label{s:implementation:stabilize}
	To stabilize the system we need to solve \cref{eq:method:stabilizationEquation}. We first derive the matrix $L$ algebraically. We then show how one can create matrix $L$ and the vector $\vec{r}$ in \cref{eq:method:stabilizationEquation}.

	%Derive matrix
	If we apply \cref{eq:single:force:simple} to compute the force on particle \freeParticle{0} in \cref{fig:implementation:intitialGrid} we find:
	\begin{equation}\label{eq:implementation:forceOnVo}
		\begin{split}
		F_{\freeParticle{0}} 	& = - [				\spring{0} \left(\fixedParticle{0} - \freeParticle{0}\right) +
													\spring{3} \left(\fixedParticle{3} - \freeParticle{0}\right) + \\
								& \quad\quad\;\;	\spring{4} \left(\freeParticle{1} - \freeParticle{0}\right)  +
													\spring{7} \left(\freeParticle{3} - \freeParticle{0}\right) ].
		\end{split}
	\end{equation}	
	Which can be rewritten as:
	\begin{equation}\label{eq:implementation:forceOnVoVersimpled}
		\begin{split}
		\spring{0}\fixedParticle{0} + \spring{3}\fixedParticle{3} 
			&= - \spring{4}\freeParticle{1} - \spring{7}\freeParticle{3} + \\
			& \quad\: \freeParticle{0} \left( \spring{0} + \spring{3} + \spring{4} + \spring{7} \right).
		\end{split}
	\end{equation}
	Based on the structure of \cref{eq:implementation:forceOnVoVersimpled} we find that the matrix $L$ has the following form:
	\begin{equation}\label{eq:implementation:rhs}
	{
		\arraycolsep=1pt
		L = \begin{bmatrix}
			\spring{0 + 3 + 4 + 7}	& -\spring{4}			& \cdots & 0\\
			- \spring{4} 			& \spring{1 + 4+ 5 + 8}	& \cdots & 0\\
			\vdots					& \vdots 				& \ddots & \vdots\\
			0						& 0 					& \cdots & \spring{16 + 19 + 20 + 23}\\
		\end{bmatrix}
	}
	\end{equation}
	where $\spring{a + b} = \spring{a} + \spring{b}$. 

	% LHS
	We build the matrix $L$ as follows
	\begin{equation}\label{eq:implementation:LHS}
		L = A^T K A .* C,
	\end{equation}
	where $A$ is the adjacency matrix, $K$ is the spring constant matrix and $C$ is the correction matrix. In \cref{fig:implementation:intitialGrid} the adjacency matrix $A$ is:
	\begin{equation}\label{eq:implementation:adjacency}
		A = \begin{bmatrix}
			1 		& 0 		& 0 		& \cdots 	& 0\\
			0 		& 1 		& 0 		& \cdots 	& 0\\
			0 		& 0 		& 1 		& \cdots 	& 0\\
			1 		& 0 		& 1 		& \cdots 	& 0\\
			1 		& 1 		& 1 		& \cdots 	& 0\\
			\vdots	& \cdots 	& \cdots 	& \ddots 	& \vdots\\
			0 		& 0 		& 0 		& \cdots 	& 1\\
		\end{bmatrix}
	\end{equation}
	The matrix $A$ has a many rows and columns as there are free particles in the grid. An element in $A$ is set to one if two particles are connected by a spring. 
	% 
	The spring constant matrix is a diagonal matrix with the spring constants of the springs in the system on the diagonal:
	\begin{equation}\label{eq:implementation:springConstantMatrix}
		K = \begin{bmatrix}
			\spring{0} 	& 0 			& \cdots & 0\\
			0 			& \spring{1}	& \cdots & 0\\	
			\vdots 		& \vdots		& \ddots & \vdots\\
			0 			& \cdots		& \cdots & \spring{23}\\
		\end{bmatrix}.
	\end{equation}	
	Lastly we need to correct the signs of the elements in the matrix, in such a way that all elements not on the diagonal are negative. We do this with the correction matrix:
	\begin{equation}\label{eq:implementation:correctionMatrix}
		C = \begin{bmatrix}
			1 			& -1 			& \cdots & -1\\
			-1 			& 1				& \cdots & -1\\	
			\vdots 		& \vdots		& \ddots & \vdots\\
			-1 			& -1			& \cdots & 1\\		
		\end{bmatrix}.
	\end{equation}

	% RHS
	\todo[inline]{Hoe komen we aan de RHS}


	\todo[inline]{Noemen welke lib, waarom die, wat we ervan gebruikt hebben.}


\subsection{Break Springs}
\label{s:implementation:break}
\todo[inline]{Hoe werkt break: denk dat deze sectie eruit gaat, zo boeiend is break niet.}

% \begin{figure}
% 	\centering
% 	\resizebox{0.9\columnwidth}{!}{%
% 		\stabilizedInitialGrid
% 	}
% 	\caption{Caption here}
% 	\label{fig:2:}
% \end{figure}

% \begin{figure}
% 	\centering
% 	\resizebox{0.9\columnwidth}{!}{%
% 		\cutGrid
% 	}
% 	\caption{Caption here}
% 	\label{fig:3:}
% \end{figure}

% \begin{figure}
% 	\centering
% 	\resizebox{0.9\columnwidth}{!}{%
% 		\stabilizedCutGrid
% 	}
% 	\caption{Caption here}
% 	\label{fig:4:}
% \end{figure}